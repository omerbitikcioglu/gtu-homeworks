\documentclass[a4 paper]{article}
\usepackage[inner=2.0cm,outer=2.0cm,top=2.5cm,bottom=2.5cm]{geometry}
\usepackage{setspace}
\usepackage[ruled]{algorithm2e}
\usepackage[rgb]{xcolor}
\usepackage{verbatim}
\usepackage{subcaption}
\usepackage{amsgen,amsmath,amstext,amsbsy,amsopn,tikz,amssymb,tkz-linknodes}
\usepackage{fancyhdr}
\usepackage[colorlinks=true, urlcolor=blue,  linkcolor=blue, citecolor=blue]{hyperref}
\usepackage[colorinlistoftodos]{todonotes}
\usepackage{rotating}
\usepackage{booktabs}
\usepackage{enumitem}
\newlist{steps}{enumerate}{1}
\setlist[steps, 1]{label = Step \arabic*:}
\newcommand{\ra}[1]{\renewcommand{\arraystretch}{#1}}

\newtheorem{thm}{Theorem}[section]
\newtheorem{prop}[thm]{Proposition}
\newtheorem{lem}[thm]{Lemma}
\newtheorem{cor}[thm]{Corollary}
\newtheorem{defn}[thm]{Definition}
\newtheorem{rem}[thm]{Remark}
\numberwithin{equation}{section}

\newcommand{\homework}[6]{
   \pagestyle{myheadings}
   \thispagestyle{plain}
   \newpage
   \setcounter{page}{1}
   \noindent
   \begin{center}
   \framebox{
      \vbox{\vspace{2mm}
    \hbox to 6.28in { {\bf CSE 211:~Discrete Mathematics \hfill {\small (#2)}} }
       \vspace{6mm}
       \hbox to 6.28in { {\Large \hfill #1  \hfill} }
       \vspace{6mm}
       \hbox to 6.28in { {\it Instructor: {\rm #3} \hfill Name: \"{O}mer Bitik\c{c}io\u{g}lu  {\rm #5} \hfill Student Id: 161044010 {\rm #6}} \hfill}
       \hbox to 6.28in { {\it Assistant: #4  \hfill #6}}
      \vspace{2mm}}
   }
   \end{center}
   \markboth{#5 -- #1}{#5 -- #1}
   \vspace*{4mm}
}

\newcommand{\problem}[2]{~\\\fbox{\textbf{Problem #1}}\hfill (#2 points)\newline\newline}
\newcommand{\subproblem}[1]{~\newline\textbf{(#1)}}
\newcommand{\D}{\mathcal{D}}
\newcommand{\Hy}{\mathcal{H}}
\newcommand{\VS}{\textrm{VS}}
\newcommand{\solution}{~\newline\textbf{\textit{(Solution)}} }

\newcommand{\bbF}{\mathbb{F}}
\newcommand{\bbX}{\mathbb{X}}
\newcommand{\bI}{\mathbf{I}}
\newcommand{\bX}{\mathbf{X}}
\newcommand{\bY}{\mathbf{Y}}
\newcommand{\bepsilon}{\boldsymbol{\epsilon}}
\newcommand{\balpha}{\boldsymbol{\alpha}}
\newcommand{\bbeta}{\boldsymbol{\beta}}
\newcommand{\0}{\mathbf{0}}


\begin{document}
\homework{Homework \#4}{Due: 27/12/19}{Dr. Zafeirakis Zafeirakopoulos}{Gizem S\"ung\"u, Ba\c{s}ak Karaka\c{s}}{}{}
\textbf{Course Policy}: Read all the instructions below carefully before you start working on the assignment, and before you make a submission.
\begin{itemize}
\item It is not a group homework. Do not share your answers to anyone in any circumstance. Any cheating means at least -100 for both sides. 
\item Do not take any information from Internet.
\item No late homework will be accepted. 
\item For any questions about the homework, send an email to gizemsungu@gtu.edu.tr
\item Submit your homework into Assignments/Homework4 directory of the CoCalc project CSE211-2019-2020.
\end{itemize}

\problem{1: Nonhomogeneous Linear Recurrence Relations}{15+15=30}
Consider the nonhomogeneous linear recurrence relation $a_n$ = 3$a_{n-1}$ + $2^n$ .\\
\subproblem{a} Show that whether $a_n$ = $-2^{n+1}$ is a solution of the given recurrence relation or not. Show your work step by step.\\
\solution
\begin{steps}
	\item $a_n=a_n^{(h)}+a_n^{(p)}$
	\item Char eq: $r-3=0$, so $r=3$
	\item $a_n^{(h)}=\alpha3^n$ \textcircled{1}
	\item $a_n^{(p)}=C.2^n$, substitute this to the recurrence relation.
	\item $C.2^n=3C.2^{n-1}+2^n$, divide both sides to $2^{n-1}$
	\item $2C=3C+2$, so $C=-2$
	\item Thus, $a_n^{(p)}=-2.2^n=-2^{n+1}$ \textcircled{2}
	\item If we combine \textcircled{1} and \textcircled{2}, $a_n=\alpha3^n-2^{n+1}$
	\item So, just $-2^{n+1}$ is not the solution.
\end{steps}

\subproblem{b} Find the solution with $a_0$ = 1.\\
\solution
\begin{steps}
	\item $a_n=\alpha3^n-2^{n+1}$, and $a_0=1$.
	\item $a_0=\alpha-2=1$, thus $\alpha=3$.
	\item $a_n=3.3^n-2^{n+1}=3^{n+1}-2^{n+1}$
\end{steps}

\problem{2: Linear Recurrence Relations}{35}
Find all solutions of the recurrence relation $a_n$ = 7$a_{n-1}$ - 16$a_{n-2}$ + 12$a_{n-3}$ + n$4^n$ with $a_0$ = -2, $a_1$ = 0, and $a_2$ = 5.\\
\solution
\begin{steps}
	\item $a_n=a_n^{(h)}+a_n^{(p)}$
	\item Char eq: $r^3-7r^2+16r-12=0$\\
	We can observe that 2 is one of the roots. \\
	This means that the equation above can be divided by (r-2).\\
	$r^3-7r^2+16r-12=(r-2)(r^2-5r+6)=(r-2)^2(r-3)$\\
	$r=2,2,3$
	\item $a_n^{(h)}=\alpha_1.2^n+\alpha_2.n.2^n+\alpha_3.3^n$ \textcircled{1}
	\item $a_n^{(p)}=C.n4^n$, substitute this to the recurrence relation.\\
	$C.n4^n=7C.n4^{n-1}-16Cn4^{n-2}+12Cn4^{n-3}+n4^n$, divide this equation to n.\\
	$C4^n=7C4^{n-1}-16C4^{n-2}+12C4^{n-3}+4^n$, divide this equation to $4^{n-3}$\\
	$64C=112C-64C+12C+64$, so $C=16$
	\item $a_n^{(p)}=16n4^n=n4^{n+2}$ \textcircled{2}
	\item If we combine \textcircled{1} and \textcircled{2}, $a_n=\alpha_1.2^n+\alpha_2.n.2^n+\alpha_3.3^n+n4^{n+2}$
	\item $a_0=\alpha_1+\alpha_3=-2$\\
	$a_1=2\alpha_1+2\alpha_2+3\alpha_3=-64$\\
	$a_2=4\alpha_1+8\alpha_2+9\alpha_3=-507$
	\item Multiply $a_1$ with $-4$ and add to $a_2$\\
	$4\alpha_1+3\alpha_3=251$\\
	$\alpha_1+\alpha_3=-2$\\
	$\alpha_1=257, \alpha_3=-259, \alpha_2=\frac{199}{2}$
	\item $a_n=257.2^n+\frac{199}{2}.n.2^n-259.3^n+n4^{n+2}$
\end{steps}

\problem{3: Linear Homogeneous Recurrence Relations }{20+15 = 35}
Consider the linear homogeneous recurrence relation $a_n$ = 2$a_{n-1}$ - 2$a_{n-2}$.
\subproblem{a} Find the characteristic roots of the recurrence relation.\\
\solution
\begin{steps}
	\item $r^n=2r^{n-1}-2r^{n-2}$
	\item $r^2-2r+2=0$
	\item $\Delta = b^2-4ac = -4$
	\item $X_{1,2}=\frac{-b\pm \sqrt{\Delta }}{2.a}
=\frac{2\pm \sqrt{-4 }}{2}=\frac{2\pm 2i}{2} $
	\item $X_{1,2} = 1+i, 1-i$
\end{steps}
\subproblem{b} Find the solution of the recurrence relation with $a_0$ = 1 and $a_1$ = 2.\\
\solution
\begin{steps}
	\item $a_n=\alpha(1+i)^n + \beta(1-i)^n$
	\item $a_0=\alpha+\beta=1$
	\item $a_1=\alpha(1+i)+\beta(1-i)=2$
	\item Multiply $a_0$ with i-1 to get rid of $\beta$
	\item $a_0=\alpha(i-1)+\beta(i-1)=i-1$
	\item $a_0+a_1=\alpha(2i)=i+1$ , so $\alpha=\frac{i+1}{2i}$
	\item Put $\alpha$ value to the equation \textcircled{2}, so $\beta=\frac{i-1}{2i}$
	\item So, $a_n=(\frac{i+1}{2i})(1+i)^n+(\frac{i-1}{2i})(1-i)^n$
\end{steps}
\end{document} 


